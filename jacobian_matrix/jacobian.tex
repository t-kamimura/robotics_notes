\documentclass[a4j,10pt]{jsarticle}

%================================================================
% Include required packages
\usepackage[dvipdfmx]{graphicx}
\usepackage{ascmac,setspace}
\usepackage{cite}
\usepackage[normalem]{ulem}
% \usepackage{fancybox}
%================================================================
% Original command definition
\def\vctr#1{\mbox{\boldmath $#1$}}
\newcommand{\bbn}[2]{\frac{{\rm {\rm d}} #1}{{\rm {\rm d}} #2}}
\newcommand{\henbbn}[2]{\frac{\partial #1}{\partial #2}}
\usepackage{amsmath, amssymb,amsfonts, amsthm}
\theoremstyle{plain}
\newtheorem{thm}{定理}
\newtheorem*{thm*}{定理}
\newtheorem*{Proof}{証明}
\newtheorem*{Proof*}{証明}
%================================================================
% Re-define commands
\makeatletter % プリアンブルで定義開始
\renewcommand{\figurename}{Fig. }  % 表示文字列を"図"から"Figure"へ
\makeatother % プリアンブルで定義終了
%================================================================
\begin{document}
%================================================================
\title{解説:ベクトルにかかっている行列を求める}
\author{上村知也}
%================================================================
\setlength{\baselineskip}{3.8mm}	% 行間の設定
\maketitle
%\thispagestyle{empty}
%\pagestyle{empty}
\setstretch{1.0} % ページ全体の行間を設定
%================================================================
% \tableofcontents
% \newpage
%================================================================


\begin{thm*}
    あるベクトル$x \in \mathbb{R}^n$, $y \in \mathbb{R}^m$の間に,行列$A \in \mathbb{R}^{m \times n}$が存在して,
    \begin{align}
        y=Ax
    \end{align}
    という関係が成立している.
    $x$と$y$は既知であり,$A\in \mathbb{R}^{m\times n}$が未知の場合を考えよう.

    このとき,ベクトル$x$にかかっている行列$A$はヤコビ行列を用いて

    \begin{align}
        A = \frac{\partial y}{\partial x}
        =\begin{bmatrix}
        \dfrac{\partial y_1}{\partial x_1} & \dfrac{\partial y_1}{\partial x_2} & \cdots & \dfrac{\partial y_1}{\partial x_n}\\
        \dfrac{\partial y_2}{\partial x_1} & \dfrac{\partial y_2}{\partial x_2} & \cdots & \dfrac{\partial y_2}{\partial x_n}\\
        \vdots & \vdots & \ddots & \vdots\\
        \dfrac{\partial y_m}{\partial x_1} & \dfrac{\partial y_m}{\partial x_2} & \cdots & \dfrac{\partial y_m}{\partial x_n}\\
        \end{bmatrix}
    \end{align}

    で求められる.
\end{thm*}

\begin{proof}[証明]
$x \in \mathbb{R}^n$に対して定義されるスカラー関数$f(x) \in \mathbb{R}$の全微分は
\begin{align}
    \mathrm{d}f(x) = \dfrac{\partial f}{\partial x_1} \mathrm{d}x_1+ \dfrac{\partial f}{\partial x_2} \mathrm{d}x_2 + \cdots +\dfrac{\partial f}{\partial x_n} \mathrm{d}x_n
\end{align}
で与えられる.

ベクトル関数$y=f(x)\in \mathbb{R}^m$でも同様に,全微分は
\begin{equation}
    \begin{aligned}
        \mathrm{d}y_1 &= f_1'(x) = \dfrac{\partial f_1}{\partial x_1} \mathrm{d}x_1+ \dfrac{\partial f_1}{\partial x_2} \mathrm{d}x_2 + \cdots +\dfrac{\partial f_1}{\partial x_n} \mathrm{d}x_n\\
    \mathrm{d}y_2 &= f_2'(x) = \dfrac{\partial f_2}{\partial x_1} \mathrm{d}x_1+ \dfrac{\partial f_2}{\partial x_2} \mathrm{d}x_2 + \cdots +\dfrac{\partial f_2}{\partial x_n} \mathrm{d}x_n\\
    &\vdots\\
    \mathrm{d}y_m &=f_m'(x) = \dfrac{\partial f_m}{\partial x_1} \mathrm{d}x_1+ \dfrac{\partial f_m}{\partial x_2} \mathrm{d}x_2 + \cdots +\dfrac{\partial f_m}{\partial x_n} \mathrm{d}x_n
    \end{aligned}
    \label{eq:dy}
\end{equation}
で与えられる.

ここでヤコビ行列$J$を
\begin{align}
    J = \frac{\partial y}{\partial x}= \frac{\partial f(x)}{\partial x}
    =\begin{bmatrix}
    \dfrac{\partial f_1}{\partial x_1} & \dfrac{\partial f_1}{\partial x_2} & \cdots & \dfrac{\partial f_1}{\partial x_n}\\
    \dfrac{\partial f_2}{\partial x_1} & \dfrac{\partial f_2}{\partial x_2} & \cdots & \dfrac{\partial f_2}{\partial x_n}\\
    \vdots & \vdots & \ddots & \vdots\\
    \dfrac{\partial f_m}{\partial x_1} & \dfrac{\partial f_m}{\partial x_2} & \cdots & \dfrac{\partial f_m}{\partial x_n}\\
\end{bmatrix}
\end{align}
とおいたとき,\eqref{eq:dy}は
\begin{align}
    \mathrm{d}y = \mathrm{d}f(x) = Jdx
\end{align}
と書ける.

一方,$y=Ax$が成立することから
\begin{align}
    \mathrm{d}y = A\mathrm{d}x
\end{align}
も成立する.
したがって,$A=J=\partial y/\partial x$となる.
\end{proof}
% ## 証明



\end{document}